\chapter{计算几何}

本章是计算几何算法的模板。若未特别说明,算法一般是~$O(1)$~的。

%%%%%%%%%%%%%%%%%%%%%%%%%%%%%%%%%%%%%%%%%%%%%%%%%%%%%%%%%%%%%%%%%%%%%%%%%%%%%%%

\section{基础算法与数据结构}

\subsection{精度控制}
\label{cgeom:precision}

这些函数使用精度控制来比较浮点数之间的关系,\emph{几乎所有}计算几何算法的实现都依赖这些函数。

\lstinputlisting{../include/cgeom/precision.h}

%%%%%%%%%%%%%%%%%%%%%%%%%%%%%%%%%%%%%%%%%%%%%%%%%%%%%%%%%%%%%%%%%%%%%%%%%%%%%%%

% \section{Data Structures}

\subsection{点的定义}
\label{cgeom:pointdef}

\lstinputlisting{../include/cgeom/point.h}

\subsection{直线、线段定义}

\lstinputlisting{../include/cgeom/line.h}

\subsection{圆形定义}

\lstinputlisting{../include/cgeom/circle.h}

\subsection{简单多边形定义}

简单多边形的定义事实上就是点的序列,按照逆时针方向依次列出点即可。或者说,多边形的定义如下:

\begin{lstlisting}
typedef vector<point> polygon;
\end{lstlisting}

%%%%%%%%%%%%%%%%%%%%%%%%%%%%%%%%%%%%%%%%%%%%%%%%%%%%%%%%%%%%%%%%%%%%%%%%%%%%%%%

\subsection{工具函数}
\label{cgeom:utility}

某些算法的实现可能会用到这些函数。

\lstinputlisting{../include/cgeom/utility.h}

%%%%%%%%%%%%%%%%%%%%%%%%%%%%%%%%%%%%%%%%%%%%%%%%%%%%%%%%%%%%%%%%%%%%%%%%%%%%%%%

\section{点、直线、线段的关系}

\subsection{两点间关系}

这个功能在~\identif{point}~结构重载的~\identif{operator==}~函数中已经实现,参见~
\ref{cgeom:pointdef}~节(第~\pageref{cgeom:pointdef}~页)。

\subsection{点与直线的关系}
\label{cgeom:relpl}

\lstinputlisting{../include/cgeom/relpl.h}

\subsection{点与线段的关系}

该实现依赖于~\ref{cgeom:relpl}~节(第~\pageref{cgeom:relpl}~页)的~\identif{relpl}。
大多数情况下,使用~\identif{relpssimp}~就够了。如果你想知道点在线段的哪个端点上,你可以用
这个算法的完全版~\identif{relps}。

\lstinputlisting{../include/cgeom/relps.h}

\subsection{两直线间关系}

该实现依赖于~\ref{cgeom:relpl}~节(第~\pageref{cgeom:relpl}~页)的~\identif{relpl}。

\lstinputlisting{../include/cgeom/relll.h}

\subsection{直线与线段的关系}

该实现依赖于~\ref{cgeom:relpl}~节(第~\pageref{cgeom:relpl}~页)的~\identif{relpl}。

\lstinputlisting{../include/cgeom/rells.h}

\subsection{两线段间关系}
\label{cgeom:relss}

This one is complicated. I will add things about it in the future.

\lstinputlisting{../include/cgeom/relss.h}

%%%%%%%%%%%%%%%%%%%%%%%%%%%%%%%%%%%%%%%%%%%%%%%%%%%%%%%%%%%%%%%%%%%%%%%%%%%%%%%

\section{与圆形有关的关系}

\lstinputlisting{../include/cgeom/ccrel.h}

%%%%%%%%%%%%%%%%%%%%%%%%%%%%%%%%%%%%%%%%%%%%%%%%%%%%%%%%%%%%%%%%%%%%%%%%%%%%%%%

\section{与多边形有关的关系}

\subsection{判断两个三角形是否相似}

\lstinputlisting{../include/cgeom/simtrtr.h}

\subsection{点与多边形关系}

该实现依赖于~\ref{cgeom:relss}~节(第~\pageref{cgeom:relss}~页)的~\identif{relss}~
和~\identif{relssraw}。

\lstinputlisting{../include/cgeom/relppo.h}

%%%%%%%%%%%%%%%%%%%%%%%%%%%%%%%%%%%%%%%%%%%%%%%%%%%%%%%%%%%%%%%%%%%%%%%%%%%%%%%

\section{图形间相交问题}

\subsection{两直线交点}

\lstinputlisting{../include/cgeom/inpll.h}

%%%%%%%%%%%%%%%%%%%%%%%%%%%%%%%%%%%%%%%%%%%%%%%%%%%%%%%%%%%%%%%%%%%%%%%%%%%%%%%

\section{图形间距离问题}

\subsection{两点间距离}

\lstinputlisting{../include/cgeom/dispp.h}

\subsection{点与直线间距离}

\lstinputlisting{../include/cgeom/displ.h}

%%%%%%%%%%%%%%%%%%%%%%%%%%%%%%%%%%%%%%%%%%%%%%%%%%%%%%%%%%%%%%%%%%%%%%%%%%%%%%%

\section{图形面积问题}

%%%%%%%%%%%%%%%%%%%%%%%%%%%%%%%%%%%%%%%%%%%%%%%%%%%%%%%%%%%%%%%%%%%%%%%%%%%%%%%

\section{图形变换问题}

\subsection{将一点以某中心旋转某角度}

\lstinputlisting{../include/cgeom/rotpcr.h}

%%%%%%%%%%%%%%%%%%%%%%%%%%%%%%%%%%%%%%%%%%%%%%%%%%%%%%%%%%%%%%%%%%%%%%%%%%%%%%%

\section{基于矩阵的图形变换}

注意把这个文件拆分掉。

\lstinputlisting{../include/cgeom/matrans.h}

%%%%%%%%%%%%%%%%%%%%%%%%%%%%%%%%%%%%%%%%%%%%%%%%%%%%%%%%%%%%%%%%%%%%%%%%%%%%%%%

\section{凸包问题}

\subsection{Graham 算法}

计算给出的点集的凸包,最坏情况为~$O(n\log{n})$。

\subsubsection{使用迭代器的 Graham 算法}

\lstinputlisting{../include/cgeom/graham.h}

\subsubsection{使用向量的 Graham 算法}

\lstinputlisting{../include/cgeom/grahamvec.h}
