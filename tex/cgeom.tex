\chapter{Computational Geometry}

%%%%%%%%%%%%%%%%%%%%%%%%%%%%%%%%%%%%%%%%%%%%%%%%%%%%%%%%%%%%%%%%%%%%%%%%%%%%%%%

\section{Fundamental}

\subsection{Precision Controlling}
\label{cgeom:precision}

These functions is implemented to compare two float numbers using precision
controlling. \emph{Almost all} computational geometry functions depend on
these.

\lstinputlisting{../include/cgeom/precision.h}

%%%%%%%%%%%%%%%%%%%%%%%%%%%%%%%%%%%%%%%%%%%%%%%%%%%%%%%%%%%%%%%%%%%%%%%%%%%%%%%

% \section{Data Structures}

\subsection{Point Definitions}
\label{cgeom:pointdef}

\lstinputlisting{../include/cgeom/point.h}

\subsection{Line Definitions}

\lstinputlisting{../include/cgeom/line.h}

%%%%%%%%%%%%%%%%%%%%%%%%%%%%%%%%%%%%%%%%%%%%%%%%%%%%%%%%%%%%%%%%%%%%%%%%%%%%%%%

\subsection{Utility Functions}
\label{cgeom:utility}

These functions might be used in some algorithms.

\lstinputlisting{../include/cgeom/utility.h}

%%%%%%%%%%%%%%%%%%%%%%%%%%%%%%%%%%%%%%%%%%%%%%%%%%%%%%%%%%%%%%%%%%%%%%%%%%%%%%%

\section{Relationship of Points, Lines and Line Segments}

\subsection{Relationship of Two Points}

This function is implemented in the overwritten \identif{operator=} in
\identif{point}. Please refer to section \ref{cgeom:pointdef} on page
\pageref{cgeom:pointdef}.

\subsection{Relationship of One Point and One Line}
\label{cgeom:relpl}

Definition of \identif{point} and \identif{line} is required. Notice that
\identif{sgn} in section \ref{cgeom:precision} and \identif{cross} in section
\ref{cgeom:utility} were used. In the following I will not waste paper to say
these dependencies because they are so widely used and/or so easy to inspect.

\lstinputlisting{../include/cgeom/relpl.h}

\subsection{Relationship of One Point and One Line Segment}

Function \identif{relpl} in section \ref{cgeom:relpl} on page
\pageref{cgeom:relpl} is required. In most cases, \identif{relpssimp} is
enough. If you want to know which endpoint that the point is exactly at, you
should use the full version \identif{relps}.

\lstinputlisting{../include/cgeom/relps.h}

\subsection{Relationship of Two Lines}

Function \identif{relpl} in section \ref{cgeom:relpl} on page
\pageref{cgeom:relpl} is required.

\lstinputlisting{../include/cgeom/relll.h}

\subsection{Relationship of One Line and One Line Segment}

Function \identif{relpl} in section \ref{cgeom:relpl} on page
\pageref{cgeom:relpl} is required.

\lstinputlisting{../include/cgeom/rells.h}

\subsection{Relationship of Two Line Segments}
\label{cgeom:relss}

This one is complicated. I will add things about it in the future.

\lstinputlisting{../include/cgeom/relss.h}

%%%%%%%%%%%%%%%%%%%%%%%%%%%%%%%%%%%%%%%%%%%%%%%%%%%%%%%%%%%%%%%%%%%%%%%%%%%%%%%

\section{Relationship about Polygons}

\subsection{Relationship of One Point and One Polygon}

Functions \identif{relss} and \identif{relssraw} in section \ref{cgeom:relss}
on page \pageref{cgeom:relss} are required.

\lstinputlisting{../include/cgeom/relppo.h}

%%%%%%%%%%%%%%%%%%%%%%%%%%%%%%%%%%%%%%%%%%%%%%%%%%%%%%%%%%%%%%%%%%%%%%%%%%%%%%%

\section{Intersection of Shapes}

\subsection{Intersection Point of Two Lines}

\lstinputlisting{../include/cgeom/inpll.h}

%%%%%%%%%%%%%%%%%%%%%%%%%%%%%%%%%%%%%%%%%%%%%%%%%%%%%%%%%%%%%%%%%%%%%%%%%%%%%%%

\section{Distance between Shapes}

\subsection{Distance between Two Points}

\lstinputlisting{../include/cgeom/dispp.h}

\subsection{Distance between One Point and One Line}

\lstinputlisting{../include/cgeom/displ.h}

%%%%%%%%%%%%%%%%%%%%%%%%%%%%%%%%%%%%%%%%%%%%%%%%%%%%%%%%%%%%%%%%%%%%%%%%%%%%%%%

\section{Area of Shapes}

%%%%%%%%%%%%%%%%%%%%%%%%%%%%%%%%%%%%%%%%%%%%%%%%%%%%%%%%%%%%%%%%%%%%%%%%%%%%%%%

\section{Transformation of Shapes}

\subsection{Rotate One Point around Given Center for Given Angle}

\lstinputlisting{../include/cgeom/rotpcr.h}

%%%%%%%%%%%%%%%%%%%%%%%%%%%%%%%%%%%%%%%%%%%%%%%%%%%%%%%%%%%%%%%%%%%%%%%%%%%%%%%

\section{Convex Hull}

\subsection{Graham Algorithm}

Calculates convex hull of given point set in $O(n\log{n})$ worst case.

\lstinputlisting{../include/cgeom/graham.h}
